\documentclass[10pt, openright]{book}

% --- KDP 5x8 PAPERBACK CONFIGURATION ---
% 5x8 inch trim size.
% 'inner' margin is larger (0.75in) to account for the spine binding (gutter).
% 'outer' margin is smaller (0.5in) to maximize reading space.
\usepackage[paperwidth=5in, paperheight=8in, top=0.75in, bottom=0.75in, outer=0.5in, inner=0.75in]{geometry}

% --- TYPOGRAPHY ---
\usepackage[T1]{fontenc}
\usepackage[utf8]{inputenc}
\usepackage[english]{babel}
\usepackage{ebgaramond} % The Classic Serif Font (Time-honored and legible)
\usepackage{microtype}
\usepackage{chngcntr}  % Essential for professional typesetting (smooths edges)

% --- STYLING PACKAGES ---
\usepackage{titlesec}
\usepackage{fancyhdr}
\usepackage{parskip}
\usepackage{amssymb}
\usepackage{emptypage} % Prevents page numbers/headers on blank pages

% --- 1. CHAPTER STYLING (The "Books") ---
\titleformat{\chapter}[display]
  {\normalfont\centering} % Global style for chapter
  {\vspace{-1cm}\itshape\large Book \thechapter} % Label "Book I"
  {10pt} % Space between label and title
  {\Huge\bfseries} % Title style
\titlespacing*{\chapter}{0pt}{0pt}{40pt}

% --- 2. SAYING STYLING (The "Statuary" Look) ---
% We use [display] to stack the number above the title
\counterwithout{section}{chapter}
\titleformat{\section}[display]
  {\normalfont\centering} % Center everything
  {\vspace{1em}\Huge\bfseries\thesection} % THE NUMBER: Huge and Bold
  {0.3em} % Small gap between number and title
  {\large\scshape} % THE TITLE: Small Caps, slightly smaller than number
  
% Adjust spacing: {left}{top}{bottom}
\titlespacing*{\section}{0pt}{1.0ex plus 1ex minus .2ex}{1.5ex plus .2ex}

% --- 3. PAGE BREAK LOGIC ---
% CRITICAL: This command ensures a saying is never split across pages.
% If it doesn't fit, it moves to the next page entirely.
\newcommand{\sectionbreak}{\filbreak}

% --- 4. ORNAMENTS ---
\newcommand{\ornament}{
    \vspace{1.5cm}
    \begin{center}
        \large $\cdot$ $\odot$ $\cdot$
    \end{center}
    \vspace{1.5cm}
}

% --- 5. HEADER/FOOTER ---
\pagestyle{fancy}
\fancyhf{}
\fancyhead[CO]{\scshape\small The Scroll of the Living Way} 
\fancyhead[CE]{\scshape\small Yeshua the Living One}
\fancyfoot[C]{\thepage}
\renewcommand{\headrulewidth}{0pt}

\begin{document}

% --- TITLE PAGE ---
\begin{titlepage}
    \centering
    \vspace*{1.5cm}
    {\Huge \textsc{The Scroll of}\\ \textsc{The Living Way}} \\
    \vspace{0.5cm}
    {\large \textit{The 81 Sayings of Yeshua the Living One}} \\
    \vspace{2cm}
     	\textbf{A Gnostic Tao for the Children of Light} \\
    \vspace{0.5cm}
    {\small Edition 2.0 $\cdot$ December 5, 2025} \\
    \vfill
    {\small Compiled \& Prepared for the Seeker of Gnosis}
    \vspace{1cm}
\end{titlepage}

% --- COPYRIGHT PAGE ---
\newpage
\thispagestyle{empty}
\vspace*{\fill}
\begin{center}
    \small
    Copyright \copyright\ 2025 by Rez Khan \\
    All rights reserved. \\
    \vspace{0.5cm}
    \textit{The Scroll of the Living Way} \\
    First Edition
\end{center}
\vspace{1cm}
\newpage

\frontmatter
\tableofcontents

\chapter{Preface}
\textit{Yeshua spoke:}

Beloved, \\
you wander in search of what has never been lost. \\
You lift your eyes to heaven, \\
yet the Kingdom is already laid in your breath. \\
You seek the Light in scriptures and temples, \\
yet the Light waits behind your seeing.

If you would know me, \\
enter the silence where your name has not yet formed. \\
There the Father of Light whispers, \\
and the Mother of Wisdom gathers you to Herself. \\
There the two are made one, \\
and the one is shown to be none.

Do not follow me as a man follows another man. \\
Follow the One who moves within your own heart, \\
for I am that One \\
when you remember yourself. \\
And I am hidden from you \\
only when you forget.

These sayings are not commandments \\
but mirrors. \\
Look into them \\
until you see your own face shining there— \\
the face you had \\
before the world divided you.

When you make the inner like the outer, \\
the above like the below, \\
the lion like the child, \\
and the child like the unbegotten, \\
then the bridal chamber opens, \\
and you awaken as the Living One.

Come, \\
remove your old garments. \\
Step naked into the light of your own being. \\
There is nothing to fear. \\
What you take to be your shadow \\
is only the veil you have outgrown.

Let the scroll unfold within you. \\
Let the sayings ripen in your silence. \\
And when the fruit is sweet, \\
eat and become what you truly are.

\vspace{1cm}
\begin{center}
    \textit{May the Inner Sun rise in your heart.}
\end{center}
\ornament

\mainmatter

% =========================================================
% BOOK 1
% =========================================================
\chapter{The Light Before the World}
\begin{center}
    \textit{(Sayings 1--9)} \\
    \vspace{0.5cm}
    \small From the Silence, the Word. \\ From the Word, the Light.
\end{center}
\vspace{0.5cm}

\section{The Way That Is Not Seen}
The Word caught in ink is a husk; \\
the Word lived is a devouring fire. \\
The Kingdom is spread upon the earth, \\
yet few behold it. \\

Those who hunt with the eyes \\
find only shadows. \\
Those who rest in the still heart \\
find the Source \\
from which all seeing springs.

\section{The Two Become One}
The world is born of the cut in two— \\
male and female, light and shadow, life and death. \\
Yet these divisions are garments only. \\

Truly I say to you: \\
when the two are made one, \\
the lion within you lies down as a child, \\
and the child awakens \\
as the One who is not born. \\

Then the bridal chamber opens in your depths.

\section{The Poverty of the Unknowing}
People cling to the loud voices of the world. \\
They trust laws more than their own heart, \\
beliefs more than living knowing. \\
Thus they sit in a treasury, \\
yet starve. \\

One who knows the self \\
no longer hoards teachings in a scroll. \\
Their very being becomes the scroll.

\section{The Father of Light}
There is a Light that gives birth to itself— \\
unborn, unbroken, unbound. \\
From this Light you came; \\
to this Light you return when you awaken. \\

The Father bears no wrath. \\
The Source casts no shadow. \\
Only the garments of the soul \\
weave darkness.

\section{The Powers Question the Soul}
Desire whispers, ``You are mine.'' \\
Ignorance declares, ``You do not know yourself.'' \\
Anger cries, ``Strike, and be struck.'' \\
The body murmurs, ``I am all there is.'' \\

But the awakened soul smiles and replies: \\
``You never touched me. \\
You grasped only my clothing.''

\section{The Hollow Reed}
The false self is a wall of stone— \\
it resists the wind and crumbles. \\
The soul is a hollow reed— \\
it lets the Spirit breathe through it. \\

When you are empty, \\
even the storm becomes a song.

\section{The Unforced Kingdom}
The Kingdom comes by neither effort nor delay. \\
When you stop grasping at life, \\
life reveals itself as eternal. \\

When you cease seeking the Kingdom, \\
you discover you never left it. \\

What you call ``sin'' is forgetting. \\
What you call ``salvation'' is remembering.

\section{The Servant at the Feast}
Be as the servant who seeks the lowest place at the feast. \\
He does not contend, \\
yet the Master calls him forward. \\

So too the one who knows the Father \\
takes the lowest seat \\
and finds there the highest. \\

Humility is not making the self small; \\
it is seeing that the self is a shadow.

\section{The Teacher Who Does Not Teach}
I do not command belief. \\
I uncover what is hidden, \\
that you may see with your own eyes. \\

A master gives answers; \\
I give you questions— \\
that you may become what I am.

\ornament

% =========================================================
% BOOK 2
% =========================================================
\chapter{The Kingdom Within}
\begin{center}
    	\textit{(Sayings 10--18)} \\
    \vspace{0.5cm}
    \small The door is shut, yet the house is vast. \ The lamp is small, yet it lights the world.
\end{center}
\vspace{0.5cm}

\section{The Single Eye}
If the eye of the body is divided, \\
it sees only confusion. \\
If the eye of the heart is single, \\
it fills the whole body with light. \\

The Nous stands between soul and spirit— \\
a lamp lit from both sides. \\
Guard this lamp, \\
and the night cannot touch you.

\section{The Inner and the Outer}
The cup is precious for its emptiness. \\
So too the self is precious \\
when it is emptied of self. \\

Inside and outside are mirrors. \\
When they reflect without distortion, \\
the All is revealed.

\section{The Lion and the Child}
The world teaches you to be a lion— \\
to conquer, to claim, to devour. \\
But I tell you: unless the lion becomes a child, \\
you will not know the Living Source. \\

The child is open, unguarded, whole. \\
There is no image of self to defend. \\
Therefore the child steps into the Kingdom easily.

\section{The Unbinding}
What binds you is not outside you. \\
What frees you is not outside you. \\
Prison and key \\
are fashioned from the same ignorance. \\

Awaken, and you will laugh— \\
for there were never any gates.

\section{The Return to the Unborn}
Before you were a form, you were the Breath. \\
Before you were breath, you were the Light. \\
Before you were the Light, \\
you were hidden in the Source. \\

To return is not to go backwards; \\
it is to remember the place \\
that has never moved.

\section{The Silence That Speaks}
Words are nets thrown into the sea. \\
They catch small fish, \\
but the Great Fish slips through. \\

Do not hunger for many words from me. \\
Hunger for the Silence \\
in which the Word is born. \\
There you and I are one life, \\
one breath, \\
one Being.

\section{The Great Stillness}
Empty yourself of what you think you are. \\
Let the waters of the heart grow still. \\
When the mud settles by itself, \\
the bottom is revealed. \\

So too, when the soul becomes quiet, \\
the Light of the Living Source \\
shines through without effort.

\section{The Hidden Guide}
The greatest guide is the one who disappears. \\
The seeker awakens and cries: \\
``Look— \\
I have found the Way myself.'' \\

Thus I leave no trace, \\
and yet my footprints are everywhere.

\section{When the Way Is Forgotten}
When the Living Way is forgotten, \\
people polish their reputations. \\
When the heart grows dim, \\
they multiply rules. \\
When rules no longer bind, \\
they threaten with fire \\
and promise crowns of gold. \\

The further you flee from your own heart, \\
the further you stray from Life. \\
Return to the beginning--- \\
to the spark before words--- \\
and fear falls away \\
like rotted cloth.
\ornament

% =========================================================
% BOOK 3
% =========================================================
\chapter{The Garment of Silence}
\begin{center}
    \textit{(Sayings 19--27)} \\
    \vspace{0.5cm}
    \small The garment divides; the wearer unites.
\end{center}
\vspace{0.5cm}

\section{Beyond the Teachings}
Leave the holiness that needs witnesses; \\
return to the wholeness \\
that needs none. \\
Leave the righteousness that lives in the scroll; \\
return to the sight \\
that burns in the heart. \\
Leave the law that threatens from outside; \\
return to the Source \\
that speaks within you as Light. \\

What you call ``good'' and ``evil'' \\
are shadows thrown by a divided mind. \\
When the eye becomes single, \\
the heart needs no commandments.

\section{Not of This World}
The world shouts; the Way whispers. \\
The world demands belief; the Way asks you to see. \\
The world praises power; the Way dissolves it. \\

Many chase desires. \\
Few seek liberation. \\
Yet the Kingdom belongs to those \\
who loosen their grip on everything.

\section{The Face of the Invisible}
The Way is a child-like mystery— \\
seen only by those who forget themselves. \\

It has no beginning and no end. \\
It moves without moving. \\
Its face cannot be drawn, \\
yet it shines through your own face \\
when you remember who you are.

\section{The One Who Is Bent}
If you wish to be whole, let yourself be broken. \\
If you wish to be full, empty yourself. \\
If you wish to be reborn, die to your false image. \\

I do not raise the proud; \\
I raise only the one \\
who has laid the self down.

\section{The Wind Speaks Briefly}
Speak only what is true, and your words will be few. \\
The wind does not shout; \\
yet entire forests bow before it. \\

When you speak from the Living Source, \\
your words carry no force— \\
yet they move mountains.

\section{The Tower That Topples}
One who raises the self has already fallen. \\
One who displays virtue has already lost it. \\
One who demands honor has already betrayed the soul. \\

Stand in your true nature— \\
and honor flows from you without asking.

\section{The Womb of Silence}
There is a womb older than creation. \\
It is the Silence that births the All. \\
It is formless, unbounded. \\
It moves through you as breath, \\
yet it is not the breath. \\

Those who know Her \\
do not cling to life nor fear death.

\section{The Heavy Root}
The root of the tree is unseen, \\
yet it holds the whole tree upright. \\
So too the silent depth of your being \\
supports your every step. \\

Those who forget their root \\
are blown about by every wind— \\
desire, fear, anger, ignorance. \\
Remember your depth, and nothing can uproot you.

\section{The Footprints of the Master}
A master walks without leaving marks. \\
Speaking without wounding. \\
Guiding without controlling. \\

To the blind—a lamp; \\
to the lost—a doorway; \\
to the weary—rest. \\
Seeing the divine spark in all beings, \\
none are beyond help.

\ornament

% =========================================================
% BOOK 4
% =========================================================
\chapter{The Power of the Gentle}
\begin{center}
    \textit{(Sayings 28--36)} \\
    \vspace{0.5cm}
    \small The gentle are the strong. \ The yielding are the immovable.
\end{center}
\vspace{0.5cm}

\section{The Fertile Soil}
Know the strength of the sower, \\
yet keep to the softness of the soil— \\
the fertile field that receives the seed. \\

Know knowledge, yet return to innocence. \\
Thus you become the place \\
where the two become one.

\section{The Futility of Control}
Those who attempt to seize the Kingdom \\
shatter it in their own hands. \\
The Kingdom is a living seed, \\
not an empire of stone. \\

Force bruises the soil. \\
Grasping crushes the grain. \\
But the one who serves the hidden life in all things \\
finds the harvest rising of itself.

\section{The Way of Non-Violence}
Those who walk in the Way do not harm. \\
They do not conquer. \\
They do not take up the sword. \\

For every blow you strike strikes you in return. \\
Every wound you inflict \\
becomes your own garment of suffering. \\

Power gained through violence rots the soul; \\
power gained through awakening cannot be taken away.

\section{The Weapon of the Heart}
Weapons are tools of fear; \\
the heart of the awakened has no use for them. \\
Victory through harm is mourning in disguise. \\

The wise triumph by dissolving conflict, \\
not by defeating an opponent— \\
seeing no opponent at all.

\section{The Way, the Truth}
People ask, ``Show us the path.'' \\
I tell them: \textbf{I am the Way.} \\
People ask, ``Show us the doctrine.'' \\
I tell them: \textbf{I am the Truth.} \\
People ask, ``Show us immortality.'' \\
I tell them: \textbf{I am the Living One.} \\

The Way is not a map; it is the Single One walking. \\
No one comes to the Father of Light \\
except by becoming what I am.

\section{Profit and Loss}
You may read the hearts of others \\
and still be a stranger to your own. \\
You may subdue cities \\
and still be ruled by fear within. \\

To turn the gaze inward is illumination. \\
To befriend your own depth is freedom. \\

For what profit is there in gaining the ten thousand things \\
if you lose the One who beholds them?

\section{The Great River}
The River of the Father and the Mother \\
flows through all worlds, \\
pouring itself into every vessel, \\
claiming nothing as its own. \\

One who walks in this River \\
acts without self-importance, \\
gives without calculation, \\
rests without pride. \\
Thus they become transparent, \\
and the Living Source is seen through them.

\section{The Face of Peace}
Hold fast to the image of the Living One \\
and all beings come to rest in your presence. \\

Peace is not the absence of conflict; \\
it is the recognition that conflict was a dream. \\
The awakened do not persuade—they radiate.

\section{The Paradox of Power}
To let something expand, first allow it to contract. \\
To let something grow strong, first allow it to weaken. \\
A seed must vanish into dark soil, \\
a breath must empty before it fills. \\

Thus I guide by reversal, \\
teaching the soul through its own emptiness.

\ornament

% =========================================================
% INSERTED SAYING: THE SWORD OF THE SPIRIT
% (Inner, non-violent discernment that cuts illusion from truth.)

\section{The Sword of the Spirit}
I did not come to bring the sleep of comfort, \\
but the sword of awakening. \\
Do you think this sword wounds flesh? \\
It severs the false from the true. \\

The sword of the Spirit is clear attention \\
held in the Father's Light. \\
It falls between fear and the one who sees fear, \\
between the story and the silent heart that hears it. \\

Where this sword descends, \\
house divides against itself— \\
the old loyalties of the self \\
stand apart from the call of the Kingdom. \\
What is of fear falls away; \\
what is of the Living One remains. \\

Blessed is the one \\
who lets this sword pass through the heart, \\
for nothing real is lost, \\
and the true self \\
stands forth whole.

\ornament

% =========================================================
% BOOK 5
% =========================================================
\chapter{The Union of Opposites}
\begin{center}
    \textit{(Sayings 37--45)} \\
    \vspace{0.5cm}
    \small The two become one.
\end{center}
\vspace{0.5cm}

\section{The Ease of the Way}
The Way acts without effort. \\
When the heart aligns with it, desire loosens its grip. \\

But when desire rises, the soul becomes troubled. \\
Use the Light to calm the waters, \\
and they become clear again.

\section{True Virtue}
High virtue moves like breath--- \\
it does not watch itself being virtuous. \\
Lesser virtue stares into its own reflection, \\
checking whether it appears upright. \\

When the Kingdom is remembered, \\
goodness blossoms of itself. \\
When it is forgotten, \\
people invent commandments \\
to restrain their own hunger.

\section{The Ones Who Remain Whole}
Heaven remains Heaven because it does not exalt itself. \\
The earth remains earth because it does not resist its nature. \\
And the soul remains whole when it refuses to be divided.

\section{The Return}
The movement of the Way is return— \\
not to the past, but to the unborn Light within you. \\
All things rise from the Source, \\
and the Source rises from silence.

\section{The Three Seekers}
The wise hear the living word \\
and let it cut them to the heart; \\
they descend gladly into their own poverty. \\
The many hear it and weigh it like a coin, \\
spending a little, keeping a little. \\
The foolish hear it and laugh aloud, \\
mocking what they have not yet seen. \\

Yet I tell you: \\
the Light plays in them all. \\
The laughers, the traders, the broken ones--- \\
each walks toward the bridal chamber, \\
though some walk with their backs turned.

\section{The Birth of the Two}
The Source gives birth to the Father and the Mother. \\
The Two give birth to the Many. \\
The Many return to the One. \\

He who sees the One in every form and motion \\
cannot be shaken.

\section{The Gentle Overcomes}
Truly I say to you: \\
what is soft and willing \\
passes where hardness breaks. \\
Water, patient and lowly, \\
hollows the rock of ages. \\

So the soul that yields to the Living One \\
passes through every barrier \\
and cannot be imprisoned.

\section{Treasure and Self}
Fame or life—which is more precious? \\
Gain or the soul—which is more valuable? \\

The one who knows their true nature \\
cannot be seduced away from it.

\section{The Great Fulness}
The cup most full appears most empty. \\
The path most straight appears crooked. \\
The hand most skilled trembles. \\

Why? Because the world sees only surfaces,\hfill\\
and the Real dwells beneath the seen. \\

The awakened make peace with paradox \\
and so are not deceived by appearances.

\ornament

% =========================================================
% BOOK 6
% =========================================================
\chapter{The Empty Vessel}
\begin{center}
    \textit{(Sayings 46--54)} \\
    \vspace{0.5cm}
    \small The cup most full appears most empty.
\end{center}
\vspace{0.5cm}

\section{The Yoke of Peace}
When the Way is lived, the ox knows its master's stall. \\
When the Way is forgotten, the sword is drawn in the street. \\

The yoke of the world is heavy with desire. \\
My yoke is easy, and brings peace.

\section{Seeing Without Seeking}
You cross seas and deserts to seek signs, \\
yet the Sign looks out through your own eyes. \\

Without leaving your house, \\
you may behold the world within you. \\
Without straining your sight, \\
you may know the One who sees. \\

The more you chase what is outside, \\
the more you forsake the Treasure within.

\section{The Unlearning}
The scholar gathers more words each day \\
and builds a tower of opinions. \\
The seeker of the Way \\
lays one stone down each day \\
until only open sky remains. \\

She strips off garment after garment--- \\
belief, fear, reputation--- \\
until the naked Light stands forth. \\

He who lets go of all things \\
discovers that he holds the All.

\section{The Heart of the Single One}
The Single One is good to those who are good, \\
and good to those who are not good— \\
seeing only the Light within them. \\

Trusting those who trust, \\
and trusting those who do not trust— \\
for trust flows from the inner being, not theirs.

\section{Life and Death}
Those who cling to life fear death. \\
Those who know Me fear neither. \\

Death is but the undressing of the bride \\
before she enters the chamber.

\section{The Nourishing Way}
The Way gives birth, nourishes, guides, protects, \\
and does so without claiming ownership. \\

Thus the Single One completes the work and forgets the work— \\
and it remains forever.

\section{Return to the Mother}
Know the Mother of your soul, \\
and you recognize all her children. \\
Know the children as passing forms, \\
and you are drawn back to the Mother. \\

In returning to the Womb of Light \\
you are no longer wounded by the powers, \\
for you remember that you were never separate \\
from the One who birthed you.

\section{The Eye of the Needle}
The world says, ``Strive to enter the narrow gate.'' \\
But I tell you: The gate is narrow only because you carry the burden of the many. \\

It is easier for a camel to pass through the eye of a needle \\
than for the one rich in self to enter the Kingdom. \\
Become small, become single, \\
and the narrow path becomes wide as the sky.

\section{The Unshakable Root}
What is rooted in the Source cannot be uprooted. \\
What is established in the Light cannot be shaken. \\

Cultivate this root in yourself, \\
and harmony spreads to all around you \\
without your speaking a word.

\ornament

% =========================================================
% BOOK 7
% =========================================================
\chapter{The Unforced Life}
\begin{center}
    \textit{(Sayings 55--63)} \\
    \vspace{0.5cm}
    \small The unforced is the natural.
\end{center}
\vspace{0.5cm}

\section{The Child of the Light}
One who lives in the Light is like a newborn child— \\
unafraid of serpents, unmoved by loud voices, unharmed by illusions. \\
The bones are soft, yet the strength is great.

\section{The One Who Knows}
The Spirit speaks only when the tongue is still. \\
The empty vessel echoes loudly; the full vessel is silent. \\
Yet I speak through silence \\
more clearly than all the world's voices. \\

Seal the senses. Quiet the mind. \\
Be still—and you behold the All.

\section{Non-Interference}
When rulers trust only decrees, \\
hearts grow clever in hiding. \\
When rulers trust only the sword, \\
fear multiplies in the streets. \\

Transform the world \\
by letting the Light transform you. \\
Cleanse the spring, \\
and the water flowing outward becomes sweet.

\section{The Mystery of Opposites}
When rulers are gentle, people thrive. \\
When rulers are harsh, people grow cunning. \\

Misfortune hides in good fortune; \\
good fortune hides in misfortune. \\
The wise trust neither shadow, but cling only to the Real.

\section{The Economy of Spirit}
Restrain desire. Conserve energy. Return to the root. \\
Those who spend life strengthening the inner being \\
become inexhaustible.

\section{Governing by Non-Grasping}
Guide the soul as you would leaven the dough: \\
gently, until the whole loaf rises. \\
The Single One governs by not ruling, \\
teaches by not preaching, \\
achieves by not striving. \\
Thus the world settles in such presence.

\section{The Great Acceptance}
The Kingdom is like the Great Sea. \\
It refuses no river, \\
no matter how bitter or clouded. \\
It judges no stream, \\
but takes all into its depth, \\
making them one. \\

So too the Living One \\
receives the pure and the stained alike, \\
until all are cleared in the vastness of Love.

\section{The Treasure Within}
The Living Way is shelter for all beings--- \\
a home for the whole and the broken alike. \\

Beautiful words can stir the mind, \\
but a heart that has found this hidden refuge \\
quietly transforms the world.

\section{The Small Actions}
Do the great work through small acts. \\
Respond to hatred with peace. \\
Untie the knot while it is loose. \\
Tend the vine while it is young. \\

The one who refuses to harm cannot be harmed.

\ornament

% =========================================================
% BOOK 8
% =========================================================
\chapter{The Wisdom of the Child}
\begin{center}
    \textit{(Sayings 64--72)} \\
    \vspace{0.5cm}
    \small The child is open, unguarded, whole.
\end{center}
\vspace{0.5cm}

\section{The First Step}
A great tree grows from a mustard seed \\
hidden in the dark earth. \\
A house is raised from a single stone. \\
The journey to the Father \\
begins exactly where your feet stand now. \\

Do not wait for a holier place or time. \\
The bridal chamber is built \\
from the step you take today.

\section{Innocence as Wisdom}
The ancient sages did not attempt to enlighten the people— \\
they helped them return to simplicity. \\
The more cleverness people acquire, \\
the further they drift from the Way.

\section{Leading by Following}
The river does not boast, \\
yet the valley drinks from it. \\
The one who makes himself high \\
passes like a cloud; \\
the one who makes himself low \\
waters many fields. \\

So the Single One leads by standing beneath all, \\
bearing the weight of the weary, \\
and because there is no competition in him, \\
there is nothing in him to oppose.

\section{The Three Jewels}
I have three treasures in this world: \\
a heart that will not wound, \\
a life that delights in simplicity, \\
and a soul that refuses to stand above another. \\

Guard these, \\
and you will walk in the Light with me. \\
Lose them, \\
and you will trade the Kingdom for dust.

\section{The Peaceful Warrior}
The greatest warrior does not fight. \\
The greatest general does not stir anger. \\
The greatest victory leaves no wounds. \\

Master yourself, and the world is mastered.

\section{The Paradox of Yielding}
There is no greater misfortune than underestimating your opponent— \\
your opponent being your own deluded self. \\
Yield, and the false self collapses.

\section{A Teaching Few Understand}
My words are simple to the ear \\
yet sharp to the heart. \\
Many copy the sayings, \\
but few let them unwind their old garments. \\

Blessed is the one \\
who wears the teaching as skin \\
and lets the former self fall away like husk. \\
Such a one walks unseen, \\
yet the world is fed by their presence.

\section{The Gift of Not-Knowing}
To know that you do not know—this is clarity. \\
To think you know while living in ignorance—this is blindness. \\
The awakened remove the cataract from their own sight first.

\section{The Fear of the False Self}
When people fear only the opinions of others, their hearts shrink. \\
Return to your true nature, and fear dissolves— \\
for what can threaten the one who knows the self is eternal?

\ornament

% =========================================================
% BOOK 9
% =========================================================
\chapter{The Return to Source}
\begin{center}
    \textit{(Sayings 73--81)} \\
    \vspace{0.5cm}
    \small The return to Source is the return to Self.
\end{center}
\vspace{0.5cm}

\section{The Courage of the Way}
Daring without wisdom leads to death. \\
Wisdom without daring leads nowhere. \\
The Way gives courage that does not wound \\
and power that does not dominate.

\section{The One Who Judges}
Why fear death? \\
The Living Source alone dissolves forms. \\
Those who attempt to take the place of the eternal \\
are like a child pretending to steer a great chariot— \\
dangerous to self and others.

\section{The Burden of Excess}
People suffer because they cling to excess. \\
The Single One lives lightly, needing little. \\
Carrying no burden, fearing no loss.

\section{The Green Wood and the Dry}
When the wood is green, it is full of sap and bends. \\
When the wood is dry, it is brittle and snaps. \\
That which yields belongs to the Living One; \\
that which resists belongs to the grave. \\
The one who remains supple cannot be broken.

\section{The Winnowing Fan}
The Father's way is like the winnowing fan: \\
separating the grain from the chaff, \\
gathering the worthy and scattering the empty. \\

As the Father separates without hatred, \\
so too the Single One gives to the needy without claiming virtue.

\section{The Stone and the Corner}
The rain that softens the earth \\
outlasts the sword that scars it. \\
The tear upon the rock, returning again and again, \\
carves what the hammer cannot. \\

The stone the builders rejected \\
has become the cornerstone of the temple. \\
So too the gentle, despised by the strong, \\
uphold a world they did not claim as theirs.

\section{The End of Debts}
Even after a truce, resentment lingers. \\
But the awakened hold no accounts. \\
They see no debtor, no creditor— \\
only the One Light in many forms.

\section{The Simple Kingdom}
Imagine a small, peaceful land \\
where people taste simplicity \\
and lose the appetite for excess. \\
Such a kingdom is within you. \\
Live there, and the world outside becomes gentle.

\section{The Completion}
Truth does not dress itself in splendor. \\
Splendor often hides a lie. \\

I give what I have been given \\
and do not count the cost. \\
The more the Single One pours out, \\
the more the Source flows through him--- \\
for he draws not from his own store, \\
but from the spring that cannot be emptied. \\

So the scripture ends where the Way begins: \\
in silence, in seeing, \\
in the Light that you are.

\ornament

% --- BACK MATTER ---
\chapter{Glossary of the Inner Kingdom}
\small
\textbf{Gnosis:} Knowledge that is not intellectual or doctrinal, but direct, experiential knowing of the Divine. It is the recognition of one's own divine origin.

\textbf{The Single One (Monachos):} A term from the Gospel of Thomas for the solitary or unified seeker; one who has integrated the inner opposites and become whole.

\textbf{Nous:} The "Eye of the Heart" or spiritual intellect. It is the faculty within the human soul that perceives the Divine directly, distinct from the rational mind.

\textbf{The Bridal Chamber:} A Gnostic metaphor for the state of union where duality (male/female, human/divine, inner/outer) is dissolved and the soul is reunited with the Spirit.

\textbf{The Mother:} The feminine aspect of the Divine (often associated with the Holy Spirit in early Semitic Christian texts), representing Wisdom (Sophia), silence, and the womb of creation.

\textbf{The Father:} The masculine aspect of the Divine, representing the originating Light, the unmanifest structure, and the Will. In this scroll, Father and Mother are the two hands of the One Source.

\textbf{The Way (Tao):} The flow of the Universe; the unnameable Source that orders all things without force. In this scroll, it is identified with the Kingdom of Heaven spread upon the earth.

\chapter{Colophon}

\vspace{2cm}

\begin{center}
    \textit{This Scroll was copied in the stillness of the turning year, \\
    by a hand seeking only to disappear, \\
    that the Light might remain.}
    
    \vspace{1cm}
    
    $\odot$
    
    \vspace{1cm}
    
    \textbf{May Peace Be Upon the Reader.}
\end{center}

\vspace{3cm}

\begin{center}
\textbf{End of Manuscript}
\end{center}

\end{document}